\documentclass{article}
\usepackage{titlesec}
\usepackage{titling}
\usepackage{hyperref}
\usepackage[margin=0.75in]{geometry}
\usepackage{qrcode}

\title{R\'esum\'e}
\author{Joshua Shaffer}

\titleformat{\section}
{\large\bfseries}
{}
{0em}
{}[\titlerule]

\titlespacing{\section}
{0em}{.25em}{0.55em}

\hypersetup{ urlbordercolor={0.75 1 1} }

\newcommand{\jskill}[2]{
\noindent
\begin{minipage}[t]{.15\textwidth}
\hfill #1
\end{minipage}
\hspace{.05em}\vline\hspace{.05em}
\begin{minipage}[t]{.80\textwidth}
#2
\end{minipage}

\vspace{0.75em}}

\newcommand{\entry}[4]{
\jskill{#1}{#2

\textit{#3}

\footnotesize{#4}}
}

\newcommand{\frc}{\href{https://www.firstinspires.org/robotics/frc}{FIRST Robotics Competition}}
\newcommand{\frct}{\href{https://www.thebluealliance.com/team/2882}{Team 2882}}

\begin{document}

\begin{center}
 \begin{minipage}{2in}% TODO make width as small as the longest line.
  \begin{center}
   \mbox{\huge\bfseries\theauthor}
   \mbox{\href{mailto:joshuatshaffer@gmail.com}{joshuatshaffer@gmail.com}}
   \mbox{\href{http://joshuatshaffer.com}{joshuatshaffer.com}}
  \end{center}
 \end{minipage}
 \hfill
 \qrcode[height=4em]{http://joshuatshaffer.com/resume/}
\end{center}

\section{Education}

\noindent
BS in Computer Science from Texas A\&M University \\
GPA 2.921, expected graduation December 2018

\section{Experience}

\entry
{2017--Present}
{Desktop Software Developer}
{Texas A\&M University}
{Designed an OS simulation and a visualizer to assist in teaching Unix programming concepts.}

\entry
{2016}
{Volunteer IT Helper}
{Lucy Hill Patterson Memorial Library}
{Helped non-technical persons to use and understand digital technologies.}

\entry
{2014}
{Team captain}
{\frc\ - \frct}
{Served as leader. Mentored new members.}

\entry
{2013-2014}
{Robotics programmer}
{\frc\ - \frct}
{Assembled and programmed cRIO for both autonomous and remote control operation.}

\section{Technical Skill Highlights}

\jskill{Haskell}{I have made a \href{https://github.com/joshuatshaffer/retro-cpu-monitor/tree/master/host-haskell}{CPU monitor}, an assembler, a \href{https://github.com/joshuatshaffer/tic-tac-lambda}{tic-tak-toe game}, a \href{https://github.com/joshuatshaffer/de-bruijn-index}{lambda calculus solver}, and a parser/interpreter. This is my favorite language.}

\jskill{C++}{Wrote an OS simulation and visualizer, \href{https://github.com/joshuatshaffer/retro-cpu-monitor/tree/master/arduino/retro_cpu_monitor_sketch}{programmed an Arduino for my CPU Monitor}, wrote a \href{https://github.com/joshuatshaffer/Command-Line-Maze}{maze generator}, and used for most of my college course work.}

\jskill{\LaTeX}{I wrote \href{https://github.com/joshuatshaffer/resume}{this r\'esum\'e}, reports and memos for work, and mathematical proofs for homework.}

\jskill{Java}{Made many small games and desktop apps. Used in \frc\ and two years of high school classes.}

\jskill{C\# \& Unity3d}{Experimented with game programming and procedural generation. Made a music visualizer.}

\jskill{Unix style 

\hfill work-flow}{I use Linux as my primary OS. For most projects \href{https://github/joshuatshaffer}{I use Git} and an automated build system such as \href{https://en.wikipedia.org/wiki/Make_(software)}{Make} or \href{https://www.haskell.org/cabal/}{Cabal}. I also use pretty-printers and linters such as \href{http://clang.llvm.org/docs/ClangFormat.html}{clang-format} to help enforce code quality. I tend to stay away from monolithic IDE's as they are not as flexible.}

\jskill{Et Cetra}{I am also familiar with many other languages and tools such as: Wordpress, JavaScript, HTML, CSS, Ruby, Python, Bash, Lua, LabVIEW, MATLAB, Arduino}

\section{Focus and Passions}

\jskill{Functional 

\hfill Programming} {I was introduced to Haskell in the fall of 2016. Since then I have continued studying and have become very interested in the functional paradigm. I am most intrigued by it's benefits to code correctness, ease of maintenance, and ease of parallelization.}

\jskill{Robotics \& IoT}{I am very interested in how computing can be applied in the physical world and solve tangible problems. I also have a lifelong passion for the physical sciences and it's fun to mix the two.}

\jskill{Teaching}{I enjoy sharing these passions with others and I am often told that I have a natural talent for explaining things. I want to pursue this at some point in my career.}

\section{Achievements}

\noindent
\begin{minipage}[t]{0.33\textwidth}
Won first place at TAMUHack \\
for \href{https://devpost.com/software/midas-evi574}{our stock trading robot}.

\hfill - October 2, 2016
\end{minipage}\hfill
\begin{minipage}[t]{0.33\textwidth}
Contributed to open source \\
by \href{https://github.com/iberianpig/fusuma/pull/36}{adding config options} to \href{https://github.com/iberianpig/fusuma}{Fusuma}.

\hfill - April 22, 2017
\end{minipage}\hfill
\begin{minipage}[t]{0.33\textwidth}
Vice President of \href{http://w5ac.tamu.edu}{W5AC},  \\
the Texas A\&M Amateur Radio Club.

\hfill - since February 9, 2017
\end{minipage}

\pagenumbering{gobble}

\end{document}